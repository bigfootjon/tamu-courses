\documentclass{article}
\usepackage{amsmath,amssymb,amsthm,latexsym,paralist}

\theoremstyle{definition}
\newtheorem{problem}{Problem}
\newtheorem*{solution}{Solution}
\newtheorem*{resources}{Resources}

\newcommand{\name}[1]{\noindent\textbf{Name: #1}}
\newcommand{\honor}{\noindent On my honor, as an Aggie, I have neither
  given nor received any unauthorized aid on any portion of the
  academic work included in this assignment. Furthermore, I have
  disclosed all resources (people, books, web sites, etc.) that have
  been used to prepare this homework. \\[1ex]
 \textbf{Signature:} \underline{\hspace*{5cm}} }

 
\newcommand{\checklist}{\noindent\textbf{Checklist:}
\begin{compactitem}[$\Box$] 
\item Did you add your name? 
\item Did you disclose all resources that you have used? \\
(This includes all people, books, websites, etc. that you have consulted)
\item Did you sign that you followed the Aggie honor code? 
\item Did you solve all problems? 
\item Did you submit the pdf file resulting from your latex file 
  of your homework?
\item Did you submit (c) a hardcopy of the pdf file in class? 
\end{compactitem}
}

\newcommand{\problemset}[1]{\begin{center}\textbf{Problem Set #1}\end{center}}
\newcommand{\duedate}[2]{\begin{quote}\textbf{Due dates:} Electronic
    submission of this homework is due on \textbf{#1} on ecampus, a
    signed paper copy of the pdf file is due on \textbf{#2} at the
    beginning of class. \end{quote} }

\newcommand{\N}{\mathbf{N}}
\newcommand{\R}{\mathbf{R}}
\newcommand{\Z}{\mathbf{Z}}


\begin{document}
\problemset{10}
\duedate{4/17/2017 before 10:00am}{4/17/2017}
\name{Jonathan Janzen}
\begin{resources} Textbook, lecture nodes, slides, talking to classmates, futoshiki.org
\end{resources}
\honor

\newpage

Read Chapter 34 in our textbook. 

\begin{problem} (15 points) % Graph-Iso is in NP
Solve Exercise 34.2-1 on page 1065. 
\end{problem}
\begin{solution}
Verification of the GRAPH-ISOMORPHISM language can be done using the input graphs $G1$  and $G2$ and a map $m$ that serve as an efficient certifier: first input is $\{G1, G2\}$ and the second is the map $m$ that serves as a certificate.
In pseudo-code:
\begin{verbatim}
def A(G1, G2, m):
    if sort(m) == [1,2,...,n] then: // This takes O(V^2) to sort
        G1_mapped = (apply m to G1)
        // Checking both vertices and edges yields O(V + E):
        return (G1_mapped is the same as G2)
    else:
        return false; // m is not a valid map
\end{verbatim}
Therefore, since the two computationally intensive parts of this problem take $O(V^2)$ and $O(V+E)$ to execute, it follows that the algorithm $A$ takes $O(V^2)$ to execute. Therefore, this is a NP problem.
\end{solution}

\begin{problem} (10 points) % If NP != co-NP then NP != P. 
Exercise 34.2-10 on page 1066. [Hint: Read Chapter 34.2 and make sure you
understand the definition of co-NP.]
\end{problem}
\begin{solution}
We can solve this problem ($NP \neq coNP \rightarrow NP \neq P$) using the contrapositive $NP = P \rightarrow NP = coNP $.
The contrapositive can be proved through the following:\\
Let $A \in NP$. Assuming $NP=P$ then $A \in P$. Since $A$ is in $P$ then it can be solved in polynomial time and it follows that the complement of the problem, $coA$ can be solved in polynomial time. The complement of $A$ can be found by reversing the yes and no conditions, which adds no complexity to the problem. This means that $coA$ is still in $P$ and $NP$. Since $coA \in NP$ then the original $A$ is in $coNP$ which implies that $coNP = NP$.
\end{solution}

\begin{problem} (15 points) % SAT in P implies one can find a
		                            % satisfying assignment in P
Exercise 34.4-6 on page 1086. 
\end{problem}
\begin{solution}
Let the polynomial-time algorithm be denoted by $A$ and let an example formula be denoted $\phi$, then the following pseudo-code can be used to find assignments in polynomial time. Here we use the notation that $\phi$ takes its first argument and replaces it everywhere and returns the simplified form of phi that can be called again to perform the same operation with the next variable\\
\begin{verbatim}
def assignments(A, phi):
    if A(phi) is NO then:
        return "phi is not satisfyable"
    else:
        x = []
        for i in [1,2,..,n]:
            if A(phi(0)) is NO then:
                x[i] = 1
            else:
                x[i] = 0
            phi = phi(x[i]) // save the replacement
        return x
\end{verbatim}
Since $A$ is known to have polynomial complexity and $A$ is called $n$ times, then $assignments$ has an overall polynomial complexity.
\end{solution}

\begin{problem} (20 points)
A partial Latin square of order $n$ is an $n\times n$ array in which
each entry is either empty or contains an element from $[n] = \{1,\ldots,
n\}$. Each row and each column contains each element from $[n]$ at
most once. Colburn showed that the problem to decide whether a given
partial Latin square can be completed to a Latin square is
NP-complete. Given this fact, show that 
\begin{compactenum}[(a)]
\item the problem to decide whether a given $n\times n$ Futoshiki
  problem can be solved is NP-complete. 
\item the problem to decide whether a given $n^2\times n^2$ Sudoku 
  problem can be solved is NP-complete. 
\end{compactenum}
\end{problem}
\begin{solution} The following are both extensions to the original Latin square problem.
\begin{compactenum}[(a)]
\item Futoshiki problems are an extension of Latin squares that add a further constraint of inequalities places between tiles.\\
    The inequalities can simply be removed and a partial Latin square problem is revealed. The reduction is therefore trivial: \\
    $$ \text{CPLS} \leq_p \text{FUTOSHIKI} $$
    The Futoshiki problem is therefore NP-Complete.
\item Sudoku problems are simply a Latin square of Latin squares, and therefore still have the same complexity as the original Latin square problem, so it is NP-complete.
\end{compactenum}
\end{solution}

\begin{problem} (20 points) % 0-1 ILP is NP complete
Exercise 34.5-2 on page 1100. 
\end{problem}
\begin{solution}
This problem can be shown to be NP complete through an efficient certifier that takes as input the $m \times n$ matrix $A$ and the $m$-vector $b$ and the $n$-vector certificate $x$. The efficient certifier works by computing the matrix-vector product $Ax$ and comparing it element-wise to $b$. This will take $O(m \times n)$ time which is polynomial. Since the certifier operates in polynomial time that means that the problem is NP.
\end{solution}

\begin{problem} (20 points) % Set partition is NP complete
Exercise 34.5-5 on page 1101 [Hint: Reduce SUBSET SUM
to SET PARTITION.] 
\end{problem}
\begin{solution}
To show that this problem is NP, we create an efficient certifier that takes as input the set $S$ and a certificate $c$ that is a vector that contains values from the set $\{0,1\}$. The efficient certifier then takes elements from $S$ where the equivalent element in $c$ is a $1$ to form $A$ and performs the same process except that the equivalent element in $c$ must be a 0 and forms $A_c$. This process takes 2 passes of $c$ so the efficient certifier operates in polynomial time and thus the problem is NP.\\
Furthermore, we can express this problem using the SUBSET SUM by trying every successive target value. Using the set returned by SUBSET SUM we can construct the certificate $c$ which we can feed into the efficient certifier to determine the correctness of the solution. The reduction we are using operates in polynomial time because it tries successive target values. Since SUBSET SUM is known to be NP-complete and the reduction is polynomial, then SET PARTITION must be NP-complete as well. 
\end{solution}

Discussions on piazza are always encouraged, especially to clarify
concepts that were introduced in the lecture. However, discussions of
homework problems on piazza should not contain spoilers. It is okay to
ask for clarifications concerning homework questions if needed. 
\medskip



\goodbreak
\checklist
\end{document}
