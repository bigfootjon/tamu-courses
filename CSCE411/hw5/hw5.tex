\documentclass{article}
\usepackage{amsmath,amssymb,amsthm,latexsym,paralist}
\usepackage[table]{xcolor}

\theoremstyle{definition}
\newtheorem{problem}{P}
\newtheorem*{solution}{Solution}
\newtheorem*{resources}{Resources}

\newcommand{\name}[1]{\noindent\textbf{Name: #1}}
\newcommand{\honor}{\noindent On my honor, as an Aggie, I have neither
  given nor received any unauthorized aid on any portion of the
  academic work included in this assignment. Furthermore, I have
  disclosed all resources (people, books, web sites, etc.) that have
  been used to prepare this homework. \\[1ex]
 \textbf{Signature:} \underline{\hspace*{5cm}} }

 
\newcommand{\checklist}{\noindent\textbf{Checklist:}
\begin{compactitem}[$\Box$] 
\item Did you add your name? 
\item Did you disclose all resources that you have used? \\
(This includes all people, books, websites, etc. that you have consulted)
\item Did you sign that you followed the Aggie honor code? 
\item Did you solve all problems? 
\item Did you submit (a) your latex source file and (b) the resulting pdf file
  of your homework?
\item Did you submit (c) a hardcopy of the pdf file in class? 
\end{compactitem}
}

\newcommand{\problemset}[1]{\begin{center}\textbf{Problem Set #1}\end{center}}
\newcommand{\duedate}[2]{\begin{quote}\textbf{Due dates:} Electronic submission of .tex
    and .pdf files of this homework is due on \textbf{#1} on csnet.cs.tamu.edu, a signed paper copy
    of the pdf file is due on \textbf{#2} at the beginning of
    class. \end{quote} }

\newcommand{\N}{\mathbf{N}}
\newcommand{\R}{\mathbf{R}}
\newcommand{\Z}{\mathbf{Z}}

\begin{document}
\problemset{5}
\duedate{2/24/2017 before 10:00am}{2/24/2017}
\name{Jonathan Janzen}
\begin{resources}
Textbook, lectures, lecture slides, talking to classmates
\end{resources}
\honor
\newpage


\begin{problem}[15 points]
Solve Exercise 15.2-1 on page 378, but replace 12 by 4 in the input. 
\end{problem}
\begin{solution}
Here is my table for $s$:\\
\begin{tabular}{| c | c c c c c c |}
\hline
 & 0 & 1 & 2 & 3 & 4 & 5\\
 \hline
0 & 0 & 150 & 210 & 310 & 1560 & 2172\\
1 &  & 0 & 120 & 210 & 2310 & 1872\\
2 &  &  & 0 & 60 & 810 & 1692\\
3 &  &  &  & 0 & 1000 & 1620\\
4 &  &  &  &  & 0 & 1500\\
5 &  &  &  &  &  & 0\\
\hline
\end{tabular}\\
Using this and the $m$ table we calculate that the optimal parenthesization is:
$$(A_1(A_2(A_3(A_4A_5))))$$
\end{solution}

\begin{problem}[15 points]
Solve Exercise 15.2-5 on page 378. Use algorithm from the textbook as
a reference. 
\end{problem}
\begin{solution}
Using the diagonal matrix strategy, we need to utilize the cells left-down and right-down diagonally from the current cell. For the cell $m[i, j]$ we need to check cells in the following ranges: (let $k,p$ be the indices of cells that make the inequality hold)\\
$$\text{if } k=i \text{ then } j < p \leq n$$
$$\text{if } p=j \text{ then } 1 \leq k < i$$
This results in the following ranges $n-j$ and $k-1$, which can be combined into the final function: $R(i,j) = (n-j) + (i-1)$. We can then use this information to compute the result of the summation:
$$ \Sigma_{i=1}^n \Sigma_{j=i}^n (n - j + i - 1) $$
$$ \Sigma_{i=1}^n ((n-i+1)n - \frac{(n-i+1)(n-i+2)}{2}) $$
$$ \Sigma_{i=1}^n (\frac{2n^2-2in+2n}{2} - \frac{i^2-2in-3i+n^2+3n+2}{2}) $$
$$ \Sigma_{i=1}^n \frac{-i^2+3i+n^2-n-2}{2} $$
$$ \frac{1}{2} \Sigma_{i=1}^n -i^2+3i+n^2-n-2 $$
$$ \frac{1}{2} (-\frac{2n^3+3n^2+n}{6}+3\frac{n^2 + n}{2}+n^3-n^2-2n) $$
$$ \frac{1}{2} (-\frac{2n^3+3n^2+n}{6}+\frac{9n^2 + 9n}{6}+\frac{6n^3-6n^2-12n}{6}) $$
$$ \frac{1}{2} (\frac{4n^3-4n}{6}) $$
$$ \frac{n^3-n}{3} $$
Thus, we have matched the original right-hand side of the equation.
\end{solution}

\begin{problem}[15 points]
Solve Exercise 15.4-1 on page 396. Show your work!
\end{problem}
\begin{solution}Using the table method:\\
\begin{tabular}{| c | c c c c c c c c |}
\hline
 & 1 & 0 & 0 & 1 & 0 & 1 & 0 & 1\\
\hline
0 & \cellcolor{yellow}$\leftarrow_0$ & $\nwarrow_1$ & $\nwarrow_1$ & $\leftarrow_1$ & $\nwarrow_1$ & $\leftarrow_1$ & $\nwarrow_1$ & $\leftarrow_1$ \\
1 & \cellcolor{yellow}$\nwarrow_1$ & $\leftarrow_1$ & $\leftarrow_1$ & $\nwarrow_2$ & $\leftarrow_2$ & $\nwarrow_2$ & $\leftarrow_2$ & $\nwarrow_2$\\
0 & $\uparrow_1$ & \cellcolor{yellow}$\nwarrow_2$ & $\nwarrow_2$ & $\leftarrow_2$ & $\nwarrow_3$ & $\leftarrow_3$ & $\nwarrow_3$ & $\leftarrow_3$\\
1 & $\nwarrow_1$ & \cellcolor{yellow}$\uparrow_2$ & $\leftarrow_2$ & $\nwarrow_3$ & $\leftarrow_3$ & $\nwarrow_4$ & $\leftarrow_4$ & $\nwarrow_4$\\
1 & $\nwarrow_1$ & \cellcolor{yellow}$\uparrow_2$ & $\leftarrow_2$ & $\nwarrow_3$ & $\leftarrow_3$ & $\nwarrow_4$ & $\leftarrow_4$ & $\nwarrow_5$\\
0 & $\uparrow_1$ & $\nwarrow_2$ & \cellcolor{yellow}$\nwarrow_3$ & $\leftarrow_3$ & $\nwarrow_4$ & $\leftarrow_4$ & $\nwarrow_5$ & $\leftarrow_5$\\
1 & $\nwarrow_1$ & $\uparrow_2$ & $\uparrow_3$ & \cellcolor{yellow}$\nwarrow_4$ & \cellcolor{yellow}$\leftarrow_4$ & $\nwarrow_5$ & $\leftarrow_5$ & $\nwarrow_6$\\
1 & $\nwarrow_1$ & $\uparrow_2$ & $\uparrow_3$ & $\nwarrow_4$ & $\leftarrow_4$ & \cellcolor{yellow}$\nwarrow_5$ & $\leftarrow_5$ & $\nwarrow_6$\\
0 & $\uparrow_1$ & $\nwarrow_2$ & $\nwarrow_3$ & $\uparrow_4$ & $\nwarrow_5$ & $\leftarrow_5$ & \cellcolor{yellow}$\nwarrow_6$ & \cellcolor{yellow}$\leftarrow_6$\\
\hline
 & 1 & 0 & 0 & 1 &  & 1 & 0 &\\
\hline
\end{tabular}\\
Using the table method we can see that the longest subsequence is "100110"
\end{solution}

\begin{problem}[15 points]
Solve Exercise 15.4-2 on page 396. 
\end{problem}
\begin{solution}Trace LCS function:\\
\begin{verbatim}
def trace_lcs(c, x, y):
    i = x.length()-1
    j = y.length()-1
    result = ""
    while(i,j != 0):
        if x[i] == y[j]:
            result += x[i]
            i = i - 1
            j = j - 1
        else:
            if c[i-1,j] < c[i, j-1]:
                j = j - 1
            else:
                i = i - 1
    return result
\end{verbatim}
\end{solution}


\begin{problem}[20 points]
Solve Exercise 15.4-5 on page 397. 
\end{problem}
\begin{solution} Here is the solution in psuedo-code: (or see the paragraph after for a textual explanation)\\
\begin{verbatim}
def mono_sub(x,y):
    result = []
    for i in 1..x.length():
        q = []
        x_sub = elements from i to the end of the sequence x
        p = x_sub[0]
        for j in x_sub:
            if j >= p:
                p = j
                q.append(j)
        if q.length() > result.length():
            result = q
    return result 
\end{verbatim}

Iterate over subsequences $s_0,s_1,...s_{n-1}$ where $s_k$ is defined as the subsequence starting at element $i$ and going to the end of the list.\\
For each subsequence, iterate over each element and check if that element is greater than $p$ ($p$ is initially defined as the first element of the subsequence) if the element is greater, then append that element to a list $q$ and set $p$ to the current element. If it is not greater, then continue on to the next element. At the end of iterating over each element, if $q$ is bigger than the subsequence to return, set that list to $q$. If not, then go on to the next element.\\
This will iterate $n$ times over $n-k$ elements (where $k$ is the starting index of each subsequence), thus:
$$ n(n-k) \in O(n^2) $$
$$ n^2-nk \in O(n^2) $$
$$ n^2 \in O(n^2) $$
Which confirms that this algorithm operates in $O(n^2)$ time.
\end{solution}

\begin{problem}[20 points]
Solve Problem 15-2 on page 405. [Hint: Given a sequence $s= \langle s_1,
s_2, \ldots, s_n\rangle$, a subsequence is obtained by deleting
elements from $s$, that is, a subsequence of $s$ is of the form 
$ \langle s_{i_{1}}, s_{i_2}, \ldots, s_{i_m}\rangle$,
where the indices satisfy $1\le i_{1} < i_{2} < \cdots < i_{m} \le n$.  
Suppose the sequence is represented by an array $s$. 
Consider the
sub-arrays $s[i..j]$. Notice that $s[i,j]$ contains a palindrome of
length $\ge 2$ when $s[i]=s[j]$. Let $l[i,j]$ denote the length of a
maximum length palindrom in $s[i,j]$. Relate $l[i,j]$ to subproblems. ] 
\end{problem}
\begin{solution}
An optimal algorithm can be provided through an adaptation the LCS algorithm. By providing the LCS with the input as one sequence and the reverse of the input as the other sequence, it will produce the longest palindrome (also known as the longest common subsequence) in $O(n^2)$ time.
\end{solution}




Discussions on ecampus are always encouraged, especially to clarify
concepts that were introduced in the lecture. However, discussions of
homework problems on ecampus should not contain spoilers. It is okay to
ask for clarifications concerning homework questions if needed. 
\medskip

\goodbreak
\checklist



\end{document}





