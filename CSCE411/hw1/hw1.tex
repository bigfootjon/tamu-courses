\documentclass{article}
\usepackage{amsmath,amssymb,amsthm,latexsym,paralist}

\DeclareRobustCommand{\stirling}{\genfrac\{\}{0pt}{}}

\theoremstyle{definition}
\newtheorem{problem}{Problem}
\newtheorem*{solution}{Solution}
\newtheorem*{resources}{Resources}

\newcommand{\name}[1]{\noindent\textbf{Name: #1}}
\newcommand{\honor}{\noindent On my honor, as an Aggie, I have neither
  given nor received any unauthorized aid on any portion of the
  academic work included in this assignment. Furthermore, I have
  disclosed all resources (people, books, web sites, etc.) that have
  been used to prepare this homework. \\[1ex]
 \textbf{Signature:} \underline{\hspace*{5cm}} }

 
\newcommand{\checklist}{\noindent\textbf{Checklist:}
\begin{compactitem}[$\Box$] 
\item Did you add your name? 
\item Did you disclose all resources that you have used? \\
(This includes all people, books, websites, etc. that you have consulted)
\item Did you sign that you followed the Aggie honor code? 
\item Did you solve all problems? 
\item Did you submit the pdf file resulting from your latex file
  of your homework?
\item Did you submit (c) a hardcopy of the pdf file in class? 
\end{compactitem}
}

\newcommand{\problemset}[1]{\begin{center}\textbf{Problem Set #1}\end{center}}
\newcommand{\duedate}[2]{\begin{quote}\textbf{Due dates:} Electronic submission of .tex
    and .pdf files of this homework is due on \textbf{#1} on e-campus
    (as a turnitin assignment), a signed paper copy
    of the pdf file is due on \textbf{#2} at the beginning of
    class. \end{quote} }

\newcommand{\N}{\mathbf{N}}
\newcommand{\R}{\mathbf{R}}
\newcommand{\Z}{\mathbf{Z}}

\begin{document}
\problemset{1}
\centerline{CSCE 411 and CSCE 411H (Dr. Klappenecker) }

\duedate{1/27/2017 before 10:00am}{1/27/2017}
\name{ Jonathan Janzen}
\begin{resources} Textbook, talking to classmates
\end{resources}
\honor

\newpage%
\noindent Get familiar with \LaTeX. All exercises are from the lecture notes
(not from our textbook). \medskip

\begin{problem}
Sign up for your section on perusall.com (watch for e-mail with sign
up instruction). Read Chapter 10 on perusall.com. Make at least 5
insightful comments (these are automatically graded. Spread them out
over the chapter and do not make them too short, the system penalizes
for that). Endorse other comments if they are helpful. 
\end{problem}
\begin{solution}
Done.
\end{solution}

\noindent \textbf{Asymptotic Equality} $\mathbf{\sim}$
\begin{problem}
Exercise 10.1
\begin{solution}
(i) $f  \sim f$:\\
$lim_{n\rightarrow\infty} \frac{|f(n)|}{|f(n)|} = lim_{n\rightarrow\infty} 1 = 1$\\
(ii) $f \sim g \leftrightarrow g \sim f$\\
$lim_{n\rightarrow\infty} \frac{|f(n)|}{|g(n)|} \leftrightarrow lim_{n\rightarrow\infty} \frac{|g(n)|}{|f(n)|}$\\
Assuming that $f \sim g$, then $lim_{n\rightarrow\infty} \frac{|f(n)|}{|g(n)|} = 1$ which means that $lim_{n\rightarrow\infty} \frac{|g(n)|}{|f(n)|}$ is also true\\
(iii) $f \sim g \land g \sim h \rightarrow f \sim h$\\
$lim_{n\rightarrow\infty} \frac{|f(n)|}{|g(n)|} == 1 \land lim_{n\rightarrow\infty} \frac{|g(n)|}{|h(n)|} == 1 \rightarrow lim_{n\rightarrow\infty} \frac{|f(n)|}{|g(n)|} == 1$
If we assume $f \sim g$ and $g \sim h$, then by the transitive property, we can assume that $f \sim h$
\end{solution}
\end{problem}

\begin{problem}
Exercise 10.2
\begin{solution}
For case $f \sim 0$:\\
$lim_{n\rightarrow\infty} \frac{|f(n)|}{|0|}$ cannot possibly be 1 because division by 0 is not possible\\
For case $0 \sim f$:\\
$lim_{n\rightarrow\infty} \frac{|0|}{|f(n)|}$ cannot possibly be 1 because any value divided by 0 is always 0 (i.e. never 1)\\
\end{solution}
\end{problem}

\begin{problem}
Exercise 10.3
\begin{solution}
$n^2 + 2n \sim n^2$:\\
Use limit to check asymptotic equality:\\
$lim_{n\rightarrow\infty} \frac{|n^2 + 2n|}{|n^2|}$\\
$ = lim_{n\rightarrow\infty} \frac{|n^2 + 2n|}{|n^2|}$\\
Drop lesser order term that doesn't matter:\\
$ = lim_{n\rightarrow\infty} \frac{|n^2|}{|n^2|}$\\
$ = lim_{n\rightarrow\infty} 1 = 1$\\
Since the limit equals one, the functions are asymptotically equal.
\end{solution}
\end{problem}

\noindent \textbf{Asymptotically Tight Bound} $\mathbf{\Theta}$
\begin{problem}
Exercise 10.9
\begin{solution}
let $f(n) = 2n^3 + 3n + 2$ and $g(n) = 5n^4 + 3n^3 + 2n + 1$\\
let $f * g = 10n^7 + ...$ (for our purposes, only the highest order term matters)\\
$f * g = 10n^7 + ... \in \Theta(n^7)$
\end{solution}
\end{problem}

\begin{problem}
Exercise 10.10 (comparing sums with integrals can be handy,
cf. Appendix A of [CLRS])
\begin{solution}
$1^k + 2^k + ... + n^k = \Theta(n^{k+1})$\\
Alternative form: $\Sigma^n_{j=1} j^k = \Theta(n^{k+1})$\\
Using integrals to approx. the summation:\\
$\int_0^n x^k dx \leq \Sigma^n_{j=1} j^k \leq \int_1^{n+1} x^k dx$\\
$x^{n+1} \leq \Sigma^n_{j=1} j^k \leq x^{n+2}$\\
$\therefore \Sigma^n_{j=1} j^k \in \Theta(x^{n+1})$
\end{solution}
\end{problem}

\begin{problem}
Exercise 10.12
\begin{solution}
(a) $f(n) = (-1)^n$
Superior only looks at even values, and inferior only looks at small values\\
$lim sup_{n\rightarrow\infty} (-1)^n = 1$\\
$lim inf_{n\rightarrow\infty} (-1)^n = -1$\\
(b) $f(n) = \frac{1}{n}$\\
$lim sup_{n\rightarrow\infty} \frac{1}{n} = DNE$\\
$lim inf_{n\rightarrow\infty} \frac{1}{n} = 0$\\
(c) $f(n) = (1 + (-1)^n)n$\\
$lim sup_{n\rightarrow\infty} (1 + (-1)^n)n = lim sup_{n\rightarrow\infty} 2n = \infty$\\
$lim inf_{n\rightarrow\infty} (1 + (-1)^n)n = lim inf_{n\rightarrow\infty} (0)n = 0$\\
(d) $f(n) = (-1)^n(2n+1)/(n+1)$\\
$lim sup_{n\rightarrow\infty} (2n+1)/(n+1) = lim sup_{n\rightarrow\infty} 2/1 = 2$\\
$lim inf_{n\rightarrow\infty} (-1)(2n+1)/(n+1) = lim inf_{n\rightarrow\infty} -2/1 = -2$\\
\end{solution}
\end{problem}

\noindent \textbf{Asymptotic Upper Bound}\textbf{O} 
\begin{problem}
Exercise 10.17
\begin{solution}
Absorption Rule: $O(f(n)) + O(g(n)) = O(g(n)) \leftrightarrow f(n) \in O(g(n))$\\
This is because $f(n) \in O(g(n))$ declares that $f(n)$ is upper-bounded by $g(n)$. Therefore, in the expression $O(f(n)) + O(g(n)) = O(g(n))$ the $O(f(n))$ can be eliminated because it's upper-bounded by $g(n)$\\
Power Rule: $(f(n) + g(n))^k = O((f(n))^k) + O((g(n))^k)$\\
We can rewrite the expansion: $(f(n) + g(n))^k = \Sigma^k_{n=0} \binom{n}{k}(f(n))^{k-n}(g(n))^n$
Only at either bound of the summation ($n=0 \lor n=k$) will the expression result in highest order polynomial components, namely: $\binom{0}{k}(f(n))^k(g(n))^0 = (f(n))^k$ and $\binom{k}{k}(f(n))^0(g(n))^k = (g(n))^k$.\\
Therefore, since all the middle components (components where $0 < n < k$) are lessor order, we can eliminate them, resulting in the following: $(f(n))^k + (g(n))^k$. Adding the Big O notation: $O((f(n))^k) + O((g(n))^k)$
\end{solution}
\end{problem}

\begin{problem}
Exercise 10.27
\begin{solution} $f \in O(g) \land f \notin o(g) \rightarrow f \in \Theta(g)$\\
Contradiction: $f(n) = x^2 - x$ and $g(n) = x^2$\\
Its trivial to demonstrate $f \in O(g)$ and $f \notin o(g)$. However, the definition for Big Theta requires that $f$ be both upper-bounded AND lower-bounded by $g$. We know already that $f \in O(g)$ but $g$ does not lower-bound $f$:\\
$f \in \Omega(g)$ requires: $x^2 -x \in \Omega(x^2)$ which is not true. Contradiction.
\end{solution}
\end{problem}

\begin{problem}
Exercise 10.28
\begin{solution}
$f(x) = g(x) + O(1)$\\
$exp(f(x) = exp(g(x) + O(1))$ (Taking the exponent of both functions)\\
$exp(f(x) = exp(g(x) + O(1))$ (Apply big theta)\\
$exp(f(x) \in \Theta(exp(g(x)))$ (Apply big theta, absorption)\\
$\therefore exp(f(x)) \asymp exp(g(x))$
\end{solution}
\end{problem}





Homeworks must be typeset in \LaTeX{}. 









\goodbreak
\checklist
\end{document}
