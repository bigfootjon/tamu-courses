\documentclass{article}
\usepackage{amsmath,amssymb,amsthm,latexsym,paralist}

\theoremstyle{definition}
\newtheorem{problem}{Problem}
\newtheorem*{solution}{Solution}
\newtheorem*{resources}{Resources}

\newcommand{\name}[1]{\noindent\textbf{Name: #1}}
\newcommand{\honor}{\noindent On my honor, as an Aggie, I have neither
  given nor received any unauthorized aid on any portion of the
  academic work included in this assignment. Furthermore, I have
  disclosed all resources (people, books, web sites, etc.) that have
  been used to prepare this homework. \\[1ex]
 \textbf{Signature:} \underline{\hspace*{5cm}} }

 
\newcommand{\checklist}{\noindent\textbf{Checklist:}
\begin{compactitem}[$\Box$] 
\item Did you add your name? 
\item Did you disclose all resources that you have used? \\
(This includes all people, books, websites, etc. that you have consulted)
\item Did you sign that you followed the Aggie honor code? 
\item Did you solve all problems? 
\item Did you submit the pdf file of your homework?
\item Did you submit (b) a hardcopy of the pdf file in class? 
\end{compactitem}
}

\newcommand{\problemset}[1]{\begin{center}\textbf{Problem Set #1}\end{center}}
\newcommand{\duedate}[2]{\begin{quote}\textbf{Due dates:} Electronic submission of .pdf files of this homework is due on \textbf{#1} on ecampus, a signed paper copy
    of the pdf file is due on \textbf{#2} at the beginning of
    class. \end{quote} }

\newcommand{\N}{\mathbf{N}}
\newcommand{\R}{\mathbf{R}}
\newcommand{\Z}{\mathbf{Z}}


\begin{document}
\problemset{2}
\duedate{2/3/2017 before 10:00am}{2/3/2017}
\name{Jonathan Janzen}
\begin{resources} Textbook, lecture slides, talking to classmates
\end{resources}
\honor

\newpage


\begin{problem}[15 points]
Consider the following code to find the second largest element: 
\begin{verbatim}
largest := numbers[0];
secondLargest := null
for i=1 to numbers.length-1 do
    number := numbers[i];
    if number > largest then
        secondLargest := largest;
        largest := number;
    else
        if number > secondLargest then
            secondLargest := number;
        end;
    end;
end;
\end{verbatim}
This code was provided by someone in response to a question on
stackoverflow. (a) How many comparisons does this code make, assuming
that \verb|numbers| contains $n$ elements. (b) Give a small example
which shows that this is not optimal.
\end{problem}
\begin{solution}
(a) In the worst case scenario, the first comparison will always be false and the second comparison is always executed. This results in $T(n) = 2(n-1)$ comparisons in the worst case and $T(n) = n-1$ comparisons in the best case. The $-1$ part deriving from the fact that the first element is skipped as its the initial max value. It could therefore be said that the running time $f(n)$ of the algorithm is bounded by $T(n) = n-1$ like so: $1*T(n) \leq f(n) \leq 2*T(n)$ which is the definition for Big Theta: $\therefore T(n) \in \Theta(n)$\\
(b) An example worst-case data set is the following
$$ [10,1,2,3,4,5,6,7,8,9] $$
where the algorithm will use 2 comparisons for each iteration, resulting in $2*9 = 18$ comparisons. Solving backwards verifys the correct value for $n$:\\
$$2*(n-1) = 18$$
$$n-1 = 9$$
$$n = 10$$
\end{solution}

\begin{problem}[15 points]
Describe an algorithm in pseudocode that finds the 2nd largest element
in the least possible number of steps. Explain why your algorithm is
correct and why it finds the 2nd largest element in the least possible
number of steps. 
\end{problem}
\begin{solution}
Psuedo-code definition:\\
\begin{verbatim}
function buildTree(vals):
    if length(vals) > 1:
        left = buildTree(1st half of vals) // including the midpoint
        right = buildTree(2nd half of vals) // excluding the midpoint
        if left[0] > right[0]:
            return (left, right) // return a tuple of the form (winner, loser[])
        else:
            return (right, left)
    else:
        // Unwrap the singleton list, there is no loser
        return (vals[0], [])
        
function secondLargest(vals):
    tupleTree = buildTree(vals)
    largest = null
    curentNode = tuppleTree
    while currentNode[1] != null:
        if currentNode[0] > largest:
            largest = currentNode[0]
        currentNode = currentNode[1]
    return largest
\end{verbatim}
The function $buildTree$ builds a tournament tree in $n-1$ comparisons and then the function $secondLargest$ compares all the direct losers to the max in $\lceil log(n) \rceil - 1$ comparisons. Overall the psuedo-code matches the known lower bound of $n-1 + \lceil log(n) \rceil - 1$.
\end{solution}

\begin{problem} (20 points)
Give a $(2n-1)$ lower bound on the number of comparisons needed to
merge two sorted lists $(a_1,a_2,\ldots,a_n)$ and $(b_1,b_2,\ldots,
b_n)$ with $a_1<a_2<\cdots <a_n$ and $b_1<b_2<\cdots <b_n$. 
[Hint: Use an adversarial method. Why can't you have in general $2n-2$ or fewer 
comparisons?] 
\end{problem}
\begin{solution}

\end{solution}

\begin{problem}(15 points) 
Solve Exercise 8.1-4 on page 194 of our textbook. 
\end{problem}
\begin{solution}
\end{solution}

\begin{problem} (15 points) Consider the task of searching a sorted
  array \verb|a[1..n]| for a given element $w$. Show that any
  algorithm that accesses the array only via Perl-style three-way
  comparisons using the $<=>$ operator (where $a <=> b$ determines
  whether $a<b$, $a=b$, or $a>b$), must take $\Omega(\log n)$ steps.
\end{problem}
\begin{solution}
\end{solution}

\begin{problem}(20 points)
 Suppose that you are given a sorted list $L$ of $n$ distinct
elements
$$ x_1 < x_2 < \cdots < x_n.$$ 
You want to search the list for an element $y$ and return $\bot$ if
$y$ is not an element of $L$, and otherwise return the index $i$ if
$y=x_i$. You can only access the list using a search procedure 
\verb|search(i,j,k,y)| that returns one of the following answers: 
\begin{inparaenum}[(i)]
\item $y<x_i$,
\item $y=x_i$,
\item $x_i<y<x_j$,
\item $y=x_j$,
\item $x_j<y<x_k$,
\item $y=x_k$,
\item $y>x_k$.
\end{inparaenum}
Here it is assumed that \verb|i<j<k|. 
Use a decision tree to prove a lower bound on the number of calls to
$\verb|search|$ that are required to correctly solve the problem. 


\end{problem}










\medskip



\goodbreak
\checklist
\end{document}
