\documentclass{article}
\usepackage{amsmath,amssymb,amsthm,latexsym,paralist}

\theoremstyle{definition}
\newtheorem{problem}{P}
\newtheorem*{solution}{Solution}
\newtheorem*{resources}{Resources}

\newcommand{\name}[1]{\noindent\textbf{Name: #1}}
\newcommand{\honor}{\noindent On my honor, as an Aggie, I have neither
  given nor received any unauthorized aid on any portion of the
  academic work included in this assignment. Furthermore, I have
  disclosed all resources (people, books, web sites, etc.) that have
  been used to prepare this homework. \\[1ex]
 \textbf{Signature:} \underline{\hspace*{5cm}} }

 
\newcommand{\checklist}{\noindent\textbf{Checklist:}
\begin{compactitem}[$\Box$] 
\item Did you add your name? 
\item Did you disclose all resources that you have used? \\
(This includes all people, books, websites, etc. that you have consulted)
\item Did you sign that you followed the Aggie honor code? 
\item Did you solve all problems? 
\item Did you submit (a) your latex source file and (b) the resulting pdf file
  of your homework?
\item Did you submit (c) a hardcopy of the pdf file in class? 
\end{compactitem}
}

\newcommand{\problemset}[1]{\begin{center}\textbf{Problem Set #1}\end{center}}
\newcommand{\duedate}[2]{\begin{quote}\textbf{Due dates:} Electronic submission of .tex
    and .pdf files of this homework is due on \textbf{#1} on csnet.cs.tamu.edu, a signed paper copy
    of the pdf file is due on \textbf{#2} at the beginning of
    class. \end{quote} }

\newcommand{\N}{\mathbf{N}}
\newcommand{\R}{\mathbf{R}}
\newcommand{\Z}{\mathbf{Z}}

\begin{document}
\problemset{5}
\duedate{2/24/2017 before 10:00am}{2/24/2017}
\name{Jonathan Janzen}
\begin{resources}
Textbook, lectures, lecture slides, talking to classmates
\end{resources}
\honor
\newpage


\begin{problem}[15 points]
Solve Exercise 15.2-1 on page 378, but replace 12 by 4 in the input. 
\end{problem}
\begin{solution}
\end{solution}

\begin{problem}[15 points]
Solve Exercise 15.2-5 on page 378. Use algorithm from the textbook as
a reference. 
\end{problem}
\begin{solution}
\end{solution}

\begin{problem}[15 points]
Solve Exercise 15.4-1 on page 396. Show your work!
\end{problem}
\begin{solution}
\end{solution}

\begin{problem}[15 points]
Solve Exercise 15.4-2 on page 396. 
\end{problem}
\begin{solution}
\end{solution}


\begin{problem}[20 points]
Solve Exercise 15.4-5 on page 397. 
\end{problem}
\begin{solution}
\end{solution}

\begin{problem}[20 points]
Solve Problem 15-2 on page 405. [Hint: Given a sequence $s= \langle s_1,
s_2, \ldots, s_n\rangle$, a subsequence is obtained by deleting
elements from $s$, that is, a subsequence of $s$ is of the form 
$ \langle s_{i_{1}}, s_{i_2}, \ldots, s_{i_m}\rangle$,
where the indices satisfy $1\le i_{1} < i_{2} < \cdots < i_{m} \le n$.  
Suppose the sequence is represented by an array $s$. 
Consider the
sub-arrays $s[i..j]$. Notice that $s[i,j]$ contains a palindrome of
length $\ge 2$ when $s[i]=s[j]$. Let $l[i,j]$ denote the length of a
maximum length palindrom in $s[i,j]$. Relate $l[i,j]$ to subproblems. ] 
\end{problem}
\begin{solution}
\end{solution}




Discussions on ecampus are always encouraged, especially to clarify
concepts that were introduced in the lecture. However, discussions of
homework problems on ecampus should not contain spoilers. It is okay to
ask for clarifications concerning homework questions if needed. 
\medskip

\goodbreak
\checklist



\end{document}





