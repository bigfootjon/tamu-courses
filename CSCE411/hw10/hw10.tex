\documentclass{article}
\usepackage{amsmath,amssymb,amsthm,latexsym,paralist}

\theoremstyle{definition}
\newtheorem{problem}{Problem}
\newtheorem*{solution}{Solution}
\newtheorem*{resources}{Resources}

\newcommand{\name}[1]{\noindent\textbf{Name: #1}}
\newcommand{\honor}{\noindent On my honor, as an Aggie, I have neither
  given nor received any unauthorized aid on any portion of the
  academic work included in this assignment. Furthermore, I have
  disclosed all resources (people, books, web sites, etc.) that have
  been used to prepare this homework. \\[1ex]
 \textbf{Signature:} \underline{\hspace*{5cm}} }

 
\newcommand{\checklist}{\noindent\textbf{Checklist:}
\begin{compactitem}[$\Box$] 
\item Did you add your name? 
\item Did you disclose all resources that you have used? \\
(This includes all people, books, websites, etc. that you have consulted)
\item Did you sign that you followed the Aggie honor code? 
\item Did you solve all problems? 
\item Did you submit the pdf file resulting from your latex file 
  of your homework?
\item Did you submit (c) a hardcopy of the pdf file in class? 
\end{compactitem}
}

\newcommand{\problemset}[1]{\begin{center}\textbf{Problem Set #1}\end{center}}
\newcommand{\duedate}[2]{\begin{quote}\textbf{Due dates:} Electronic
    submission of this homework is due on \textbf{#1} on ecampus, a
    signed paper copy of the pdf file is due on \textbf{#2} at the
    beginning of class. \end{quote} }

\newcommand{\N}{\mathbf{N}}
\newcommand{\R}{\mathbf{R}}
\newcommand{\Z}{\mathbf{Z}}


\begin{document}
\problemset{10}
\duedate{4/17/2017 before 10:00am}{4/17/2017}
\name{ (put your name here)}
\begin{resources} (All people, books, articles, web pages, etc. that
  have been consulted when producing your answers to this homework)
\end{resources}
\honor

\newpage

Read Chapter 34 in our textbook. 

%\begin{problem} (15 points)
%Solve Exercise 34.1-6 on page 1061. [Hint: Carefully read Section 34.1 and make
%sure that you understand the notations.] 
%\end{problem}

\begin{problem} (15 points) % Graph-Iso is in NP
Solve Exercise 34.2-1 on page 1065. 
\end{problem}

\begin{problem} (10 points) % If NP != co-NP then NP != P. 
Exercise 34.2-10 on page 1066. [Hint: Read Chapter 34.2 and make sure you
understand the definition of co-NP.]  
\end{problem}

\begin{problem} (15 points) % SAT in P implies one can find a
                            % satisfying assignment in P
Exercise 34.4-6 on page 1086. 
\end{problem}

\begin{problem} (20 points) 
A partial Latin square of order $n$ is an $n\times n$ array in which
each entry is either empty or contains an element from $[n] = \{1,\ldots,
n\}$. Each row and each column contains each element from $[n]$ at
most once. Colburn showed that the problem to decide whether a given
partial Latin square can be completed to a Latin square is
NP-complete. Given this fact, show that 
\begin{compactenum}[(a)]
\item the problem to decide whether a given $n\times n$ Futoshiki
  problem can be solved is NP-complete. 
\item the problem to decide whether a given $n^2\times n^2$ Sudoku 
  problem can be solved is NP-complete. 
\end{compactenum}
\end{problem}

\begin{problem} (20 points) % 0-1 ILP is NP complete
Exercise 34.5-2 on page 1100. 
\end{problem}

\begin{problem} (20 points) % Set partition is NP complete
Exercise 34.5-5 on page 1101 [Hint: Reduce SUBSET SUM
to SET PARTITION.] 
\end{problem}

Discussions on piazza are always encouraged, especially to clarify
concepts that were introduced in the lecture. However, discussions of
homework problems on piazza should not contain spoilers. It is okay to
ask for clarifications concerning homework questions if needed. 
\medskip



\goodbreak
\checklist
\end{document}
