\documentclass{article}
\usepackage{amsmath,amssymb,amsthm,latexsym,paralist}

\theoremstyle{definition}
\newtheorem{problem}{Problem}
\newtheorem*{solution}{Solution}
\newtheorem*{resources}{Resources}

\newcommand{\name}[1]{\noindent\textbf{Name: #1}}
\newcommand{\honor}{\noindent On my honor, as an Aggie, I have neither
  given nor received any unauthorized aid on any portion of the
  academic work included in this assignment. Furthermore, I have
  disclosed all resources (people, books, web sites, etc.) that have
  been used to prepare this homework. \\[1ex]
 \textbf{Signature:} \underline{\hspace*{5cm}} }

 
\newcommand{\checklist}{\noindent\textbf{Checklist:}
\begin{compactitem}[$\Box$] 
\item Did you add your name? 
\item Did you disclose all resources that you have used? \\
(This includes all people, books, websites, etc. that you have consulted)
\item Did you sign that you followed the Aggie honor code? 
\item Did you solve all problems? 
\item Did you submit (a) your latex source file and (b) the resulting pdf file
  of your homework?
\item Did you submit (c) a hardcopy of the pdf file in class? 
\end{compactitem}
}

\newcommand{\problemset}[1]{\begin{center}\textbf{Problem Set
      #1}\end{center}}
\newcommand{\duedate}[2]{\begin{quote}\textbf{Due dates:} Electronic
    submission of this homework is due on \textbf{#1} on ecampus, a
    signed paper copy of the pdf file is due on \textbf{#2} at the
    beginning of class. \end{quote} }

\newcommand{\N}{\mathbf{N}}
\newcommand{\R}{\mathbf{R}}
\newcommand{\Z}{\mathbf{Z}}


\begin{document}
\problemset{7}
\duedate{Monday 3/27/2017 before 10:00am}{3/27/2017}
\name{Jonathan Janzen}
\begin{resources}
Textbook, Perusall readings, lecture slides, talking to classmates.
\end{resources}
\honor

\begin{problem}[20 points] 
On perusall.com, make at least five insightful remarks on the chapter on
probability theory. You are not allowed to give spoilers on exercise
solutions. 
\end{problem}
\begin{solution} Type in your comments in perusall.com. You need to
  spread out your comments. It is recommended that you do not make the
  comments too short. Comments are automatically graded by
  perusall.com. 
\end{solution}

\begin{problem}[10 points] 
Solve Exercise 2.2 on page 3 of the probability theory lecture notes
in perusall. 
\end{problem}
\begin{solution} If $F$ contains more outcomes than $E$, then $E-F$ will be the impossible event, which is contained in $\mathcal{F}$ since $\mathcal{F}$ all possible subsets of the sample space $\Omega$. If the opposite is true, then $E$ has more outcomes of $F$, then the result of $E-F$ is some subset of $E$. Since we already know that $E$ is contained in $\mathcal{F}$ (by definition), and $E-F$ is a subset of $E$. Thus, $E-F$ is contained in $\mathcal{F}$ because it is the set of all subsets of the sample space.
\end{solution}

\begin{problem}[10 points] 
Solve Exercise 2.4 on page 3 of the probability theory lecture notes
in perusall. 
\end{problem}
\begin{solution}
Since we know that the events are not necessary disjoint, they could share outcomes. Therefore, the union between 2 events will ignore the overlap of outcomes and result in a probability less than if the probabilities of each event are considered separately:
$$ Pr[E_1 \cup E_2] \leq Pr[E_1] + Pr[E_2] $$
Therefore, if we know that, for example, $E_1$ and $E_2$ share an outcome, then for all the events $E_1,E_2,...,E_n$ the probability of the union of the events will be less than the addition of each probability considered separately:
$$ Pr[E_1 \cup E_2 \cup ... \cup E_n] \leq Pr[E_1] + Pr[E_2] + ... + Pr[E_n] $$
This works where any amount of outcomes are shared between any 2 or more events being considered. The only requirement is that some outcomes are shared.
\end{solution}

\begin{problem}[10 points] 
Solve Exercise 2.6 on page 4 of the probability theory lecture notes
in perusall. 
\end{problem}
\begin{solution}
We can use the conditional probability rule to simplify the summation expression:
$$ \Sigma_{k=1}^n Pr[E | F_k]Pr[F_k] = \Sigma_{k=1}^n Pr[E \cap F_k] $$
The $Pr[E \cap F_k]$ part of the expression is simply taking a slice of $E$ in terms of $F_k$. Since we know that the $F_k$ events partition $\Omega$, it therefore follows that the intersect with $E$ summed up will produce $P[E]$:
$$\Sigma_{k=1}^n Pr[E \cap F_k] = Pr[E] $$
Thus, from beginning to end:
$$ \Sigma_{k=1}^n Pr[E | F_k]Pr[F_k] = \Sigma_{k=1}^n Pr[E \cap F_k] = Pr[E] $$
\end{solution}

\begin{problem}[15 points] Consider the set $S = \{1,2,\ldots, n\}$. We generate
  a subset $X$ of $S$ as follows: a fair coin is flipped independently
  for each element in $S$; if the coin lands on heads, then the
  element is added to $X$, and otherwise it is not added. Show that
  $X$ is equally likely to be any of the $2^n$ possible subsets. 
\end{problem}
\begin{solution}
Since each coin flip is independent of all other coin flips, each has an equal chance of occurring. To relate $S$ to $X$, we introduce another set $B$ which contains the results of all the coin flips. An example of $B$ could be the following:
$$ B = \{\text{head, tails, tails, heads, heads, heads, ...}\} $$
Since the coin flips all have an equal chance of occurring, each instance of $B$ has an equal chance of occurring. Each element of $S$ is only included in $X$ if the element at the matching index in $B$ is heads. This can be shown by the following:
$$ X = [e | (e, b) \in \text{zip}(S, B), b = \text{heads}] $$
Because of this, each set $X$ has an equal chance of occurring as a subset of $S$, even though its length, for one, does not have an equal distribution.
\end{solution}

\begin{problem}[15 points]
Suppose that two sets $X$ and $Y$ are chosen independently and
uniformly at random from all the $2^n$ subsets of $S=\{1,2,\ldots,
n\}$. Determine $\Pr[X\subseteq Y]$. 
\end{problem}
\begin{solution}
This problem depends on the cardinality of $Y$, so we need to sum all cases of cardinality of $Y$ (using our proof from problem 4):
$$ \Pr[X\subseteq Y] = \Sigma_{k=1}^n Pr[X \subseteq Y | |Y| = k] Pr[y = k] $$
$Pr[|Y| = k]$ is equivalent to the number of subsets of length $k$ divided by the total number of subsets. In other words:
$$ Pr[|Y| = k] = \frac{{n \choose k}}{2^n} $$
Additionally, if $Y$ has cardinality $k$, then there are $2^k$ subsets that $X$ can be, and we know that each of these are equally likely (from problem 5). These choices for the value of $X$ are denoted $X_1,X_2,X_3,...,X_{2^k}$. Therefore, $Pr[X=X_i] = \frac{1}{2^n}$ for all $i$. Thus:
$$ Pr[X \subseteq Y | |Y| = k] = Pr[U_{i=1}^{2^k} X = X_i] = \Sigma_{i=1}^{2^k}Pr[X = X_i] = \Sigma_{i=1}^{2^k}\frac{1}{2^n} = \frac{2^k}{2^n} $$
Putting that all together:
$$ Pr[X \subseteq Y] = Pr[X \subseteq Y | |Y| = k]Pr[|y| = k] = \Sigma _{k=0}^n \frac{2^k}{2^n} \frac{{n \choose k}}{2^n}$$
Performing simplification and using binomial theorem rules:
$$ = \frac{1}{2^n} \Sigma_{k=0}^{2^n} (\frac{1}{2})^{n-k} {n \choose k} = \frac{1}{2^n} (\frac{1}{2} + 1)^n = \frac{1}{2^n} (\frac{3}{2})^n = \frac{3^n}{4^n} $$
Thus, the probability that $X$ is a subset $Y$ is given by $\frac{3^n}{4^n}$
\end{solution}

\begin{problem}[20 points]
There may be several different min-cut sets in a graph with $n$ vertices. Show that
there can be at most $n(n-1)/2$ distinct min-cut sets. [Hint: The
analysis of the min-cut algorithm can help.] 
\end{problem}
\begin{solution}
Since we know that the min-cut algorithm is randomized, given enough iterations it will produce all possible outputs with a probability of $\frac{2}{n(n-1)}$. Since the probability in this case represents the reciprocal of total possibilities it is trivial to produce:
$$ \frac{1}{\frac{2}{n(n-1)}} = \frac{n(n-1)}{2}$$
Therefore, there are $\frac{n(n-1)}{2}$ max possiblities.
\end{solution}





\goodbreak
\checklist
\end{document}
