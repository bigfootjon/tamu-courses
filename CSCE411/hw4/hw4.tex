\documentclass{article}
\usepackage{amsmath,amssymb,amsthm,latexsym,paralist}

\theoremstyle{definition}
\newtheorem{problem}{Problem}
\newtheorem*{solution}{Solution}
\newtheorem*{resources}{Resources}

\newcommand{\name}[1]{\noindent\textbf{Name: #1}}
\newcommand{\honor}{\noindent On my honor, as an Aggie, I have neither
  given nor received any unauthorized aid on any portion of the
  academic work included in this assignment. Furthermore, I have
  disclosed all resources (people, books, web sites, etc.) that have
  been used to prepare this homework. \\[1ex]
 \textbf{Signature:} \underline{\hspace*{5cm}} }

 
\newcommand{\checklist}{\noindent\textbf{Checklist:}
\begin{compactitem}[$\Box$] 
\item Did you add your name? 
\item Did you disclose all resources that you have used? \\
(This includes all people, books, websites, etc. that you have consulted)
\item Did you sign that you followed the Aggie honor code? 
\item Did you solve all problems? 
\item Did you submit the pdf (derived from your latex file)
  of your homework?
\item Did you submit a hardcopy of the pdf file in class? 
\end{compactitem}
}

\newcommand{\problemset}[1]{\begin{center}\textbf{Problem Set #1}\end{center}}
\newcommand{\duedate}[2]{\begin{quote}\textbf{Due dates:} Electronic
    submission of the .pdf file of this homework is due on \textbf{#1} on ecampus, a signed paper copy
    of the pdf file is due on \textbf{#2} at the beginning of
    class. \end{quote} }

\newcommand{\N}{\mathbf{N}}
\newcommand{\R}{\mathbf{R}}
\newcommand{\Z}{\mathbf{Z}}


\begin{document}
\problemset{4}
\duedate{2/17/2017 before 10:00am}{2/17/2017}
\name{Jonathan Janzen}
\begin{resources} Textbook, Lectures/Lecture slides, talking to classmates
\end{resources}
\honor

\newpage
\noindent Make sure that you describe all solutions in your own words. Read Chapter 16.4 and skim rest of the chapter. \medskip


\begin{problem} (15 points)
Suppose that a country adopts coins of values 1, 10, and 21. Does the
greedy algorithm to give change always give the fewest number of
coins?  Prove it or give the smallest counter example. 
\end{problem}
\begin{solution}
Smallest counter example is $30$, which is because a greedy algorithm will select (in order) a 21 coin then nine 1 coins while the optimal solution is three 10 coins.
\end{solution}

\begin{problem} (15 points)
Suppose that a coin system has values $1, c_2,$ and  $c_3$ with
$1<c_2<c_3$. We say that the coin system is non-canonical if and only
if there exists an amount $x>1$ for which the greedy algorithm for
giving change does not produce the fewest number of coins (i.e., the
greedy algorithm is not optimal). Show that $(1,c_2,c_3)$ is
non-canonical if and only if the quotient $q$ and remainder $r$ of
dividing $c_3$ by $c_2$, that is, 
$c_3 = qc_2+r$ with $0\le r<c_2$, 
satisfy $0< r < c_2 - q$. \\{}
[Hint: You can freely use the fact proven by Kozen and Zaks that the 
smallest counterexample $x$ is in the range $c_3+1 < x < c_2+c_3$.
Show that the smallest counterexample $x$ uses just coins of value
$c_2$ in the optimal solution, and only coins of value $1$ and $c_3$
in the greedy solution. ]
\end{problem}
\begin{solution}
\end{solution}

\begin{problem} (10 points)
Consider Kruskals algorithm for a graph $G=(V,E)$ with edges 
$\{1,2\}$, $\{2,3\}$, $\{3,4\}$, $\{4,1\}, \{2,4\}$. If the weight of the edges
are 
$$ w(\{1,2\})=1, w(\{2,3\})=5, w(\{3,4\})=6, w(\{4,1\})=2, w(\{2,4\})=4.$$
In which order will Kruskal's algorithm pick the edges? 
\end{problem}
\begin{solution}
Kruskal's algorithm will select edges from least weight to most weight, skipping any edges that create loops and stopping when the graph is connected: $\{1,2\},\{4,1\},\{2,3\}$ (skipping $\{2,4\}$, and stopping before $\{3,4\}$)
\end{solution}

\begin{problem}(20 points) The specification of a matroid $M=(S,F)$ as
  a finite set $S$ together with a nonempty family $F$ of subsets of $S$ that
  is hereditary and satisfies the exchange axiom is not particularly
  economical. A set in $F$ that is maximal with respect to inclusion
  is called a basis of the matroid. Let $\mathcal{B}$ denote the set
  of bases of $M$. Then $(S,\mathcal{B})$ is a more economical
  way to represent the matroid $M$. Show that 
\begin{compactenum}[(a)]
\item $\mathcal{B}$ is not empty, 
\item all bases in $\mathcal{B}$ have the same cardinality,
\item the set family $\mathcal{B}$ satisfies the condition: If $B_1$
  and $B_2$ are bases in $\mathcal{B}$ and $x\in B_1\setminus B_2$,
  then there exists an element $y \in B_2\setminus B_1$ such that
  $(B_1\setminus \{x\} ) \cup \{y\}$ is a basis in $\mathcal{B}$. 
\item one can recover the matroid $(S,F)$ from $(S,\mathcal{B})$,
\item more generally, given a finite set $S$ and a family 
  $\mathcal{B}$ of subsets of $S$ satisfying (a), (b), and (c) gives
  rise to a matroid using your technique from (d). 
\end{compactenum}
\end{problem}
\begin{solution}
\begin{compactenum}[(a)]
\item $\mathcal{B}$ cannot be empty because it must have at least one base because the definition of a matroid states that $F$ cannot be empty.
\item If the sets in $\mathcal{B}$ are \textit{maximal} sets in $F$, meaning that they all have the largest cardinality yet are still subsets of $S$. If they all have the largest cardinality that still qualifies, then they must all have the same cardinality.
\item This condition can be expressed as removing an element from $B_1$ that is not in $B_2$ and adding an element that is in $B_2$ but not in $B_1$. This method takes a basis in $\mathcal{B}$ and removes its status as a basis (effectively adding it to $F$) and then adds a different element to restore its status as a basis (removing it from $F$).
\item Since the bases are maximal, the removal of elements will reproduce sets in $F$. Iterating over all the bases will provide a full family of sets that form $F$.
\item By using (d), we can transform $(S, \mathcal{B})$ into a matroid $(S, F)$
\end{compactenum}
\end{solution}


\begin{problem}(20 points)
Exercise 16.4-4 on page 443 in our textbook. 
\end{problem}
\begin{solution}
\end{solution}

\begin{problem}(20 points)
  Let $S$ be a finite set, $F$ a nonempty family of subsets of $S$ that
  satisfies the hereditary axiom. Show that if $(S,F)$ is \textbf{not}
  a matroid, that is, does not satisfy the exchange axiom, then there
  exists a weight function $w\colon S \rightarrow \mathbf{R_{\ge 0}}$
  such that Greedy($(S,F)$, $s$) does not return a maximum weight
  basis of $F$, (a basis is a set in $F$ that is not contained in any
  larger set in $F$). [Hint: Consider two subsets $A$ and $B$ such
  that $|A|<|B|$ but such that there does not exist any $x\in
  B\setminus A$ satisfying $A\cup \{x\}$ in $F$. Assume that $A$ has
  $m$ elements and construct a weight $w$ such that the algorithm will
  return a set that has weight $w(A)$ even though $w(A)<w(B)$. ]
\end{problem}
\begin{solution}
\end{solution}







Discussions on piazza are always encouraged, especially to clarify
concepts that were introduced in the lecture. However, discussions of
homework problems on piazza should not contain spoilers. It is okay to
ask for clarifications concerning homework questions if needed. 
\medskip



\goodbreak
\checklist
\end{document}
