\documentclass{article}
\usepackage{amsmath,amssymb,amsthm,latexsym,paralist,relsize}

\theoremstyle{definition}
\newtheorem{problem}{P}
\newtheorem*{solution}{Solution}
\newtheorem*{resources}{Resources}

\newcommand{\name}[1]{\noindent\textbf{Name: #1}}
\newcommand{\honor}{\noindent On my honor, as an Aggie, I have neither
  given nor received any unauthorized aid on any portion of the
  academic work included in this assignment. Furthermore, I have
  disclosed all resources (people, books, web sites, etc.) that have
  been used to prepare this homework. \\[1ex]
 \textbf{Signature:} \underline{\hspace*{5cm}} }

 
\newcommand{\checklist}{\noindent\textbf{Checklist:}
\begin{compactitem}[$\Box$] 
\item Did you add your name? 
\item Did you disclose all resources that you have used? \\
(This includes all people, books, websites, etc. that you have consulted)
\item Did you sign that you followed the Aggie honor code? 
\item Did you solve all problems? 
\item Did you submit the pdf file
  of your homework?
\item Did you submit a hardcopy of the pdf file in class? 
\end{compactitem}
}

\newcommand{\problemset}[1]{\begin{center}\textbf{Problem Set #1}\end{center}}
\newcommand{\duedate}[2]{\begin{quote}\textbf{Due dates:} Electronic submission of .tex
    and .pdf files of this homework is due on \textbf{#1} on csnet.cs.tamu.edu, a signed paper copy
    of the pdf file is due on \textbf{#2} at the beginning of
    class. \end{quote} }

\newcommand{\N}{\mathbf{N}}
\newcommand{\R}{\mathbf{R}}
\newcommand{\Z}{\mathbf{Z}}

\begin{document}
\problemset{6}
\duedate{3/6/2017 before 10:00am}{3/6/2017}
\name{Jonathan Janzen}
\begin{resources} Textbook, lectures, talking to classmates
\end{resources}
\honor
\newpage


\begin{problem}[20 points]
Exercise 17.1-1 on page 456
\end{problem}
\begin{solution}
Since MULTIPUSH depends on $k$ iterations of PUSH which operates in $O(1)$ time:
$$ k * O(1) = O(k)$$
Thus, performing MULTIPUSH $n$ times leads to $O(nk)$. Amortized:
$$ \frac{O(nk)}{n} = O(k) $$
Therefore, the $O(1)$ bound on stack operations does not hold with the inclusion of MULTIPUSH.
\end{solution}

\begin{problem}[20 points]
Exercise 17.1-2 on page 456
\end{problem}
\begin{solution}
The cost of a single DECREMENT operation is given by $\Theta(k)$. This is derived from the fact that it could either take $1$ bit operation: (let $k=8$)
$$00000001 \rightarrow 00000000$$
Or it could take $k$ bit operations:
$$10000000 \rightarrow 01111111$$
So overall it will take $\Theta(k)$. Performing $n$ operations is a simple multiplication:
$$n * \Theta(k) = \Theta(nk)$$
$\therefore$ $n$ DECREMENT operations on a $k$-bit counter will take $\Theta(nk)$ operations
\end{solution}

\begin{problem}[20 points]
Exercise 17.1-3 on page 456, but use as cost $i^2$ (instead of $i$) when $i$ is a power of $2$, and $1$ otherwise. 
\end{problem}
\begin{solution}
In the cases where $i$ is not a power of 2, the operation costs $O(1)$ so $n$ operations can be done in linear time ($n$)\\
Alternatively, when $i$ is a power of 2, the costs for $n$ operations is of the form: 
$$ \mathlarger{\mathlarger{\Sigma_{i=0}^{\text{log}_2 n}}} \frac{n}{2^i} = n + \frac{n}{2} + \frac{n}{4} + \frac{n}{8} + ... + 8 + 4 + 2 + 1 $$
Solving the summation:
$$ \mathlarger{\mathlarger{\Sigma_{i=0}^{\text{log}_2 n}}} \frac{n}{2^i} = 2n$$
Thus, the running time for cases when $i$ is a power of 2 is $2n$\\
Overall, we have:
$$ 2n + n = 3n $$
Leading to Big O complexity of $O(n)$
\end{solution}


\begin{problem}[20 points]
Exercise 17.2-1 on page 458
\end{problem}
\begin{solution}
Each push and pop operation is assigned 2 credits, copy is assigned 0 credits. The first credit (for push and pop) is used to complete the operation and the second is stored. It follows that after any set of $k$ operations (push or pop), the stack will be credited with $k$. This is due to the fact that $k$ push or pop operations generate $2k$ credits, and half of those ($k$) are used to pay for the operations, and the other half remain. The remaining credits can be used to pay for the copy, since it is assigned 0 credits. Therefore, the overall cost of creating the stack, including the copy, costs $O(n)$ given that $k$ operations take $O(k)$ cost.
\end{solution}

\begin{problem}[20 points]
Exercise 17.2-2 on page 459 (use our modified version \textbf{P3} above). 
\end{problem}
\begin{solution}
If $i$ is a power of 2, then the cost is 2, if it is not then the cost is 3.\\
Therefore, each operation runs in linear time $O(1)$, so for $n$ operations it will be:
$$ O(1) * n = O(n) $$
Amortized cost using aggregate method:
$$ \frac{O(n)}{n} = O(1) $$
$\therefore$ amortized cost is $O(1)$
\end{solution}






Discussions on ecampus are always encouraged, especially to clarify
concepts that were introduced in the lecture. However, discussions of
homework problems on ecampus should not contain spoilers. It is okay to
ask for clarifications concerning homework questions if needed. 
\medskip

\goodbreak
\checklist



\end{document}





