\documentclass{article}
\usepackage{amsmath,amssymb,amsthm,latexsym,paralist}

\theoremstyle{definition}
\newtheorem{problem}{Problem}
\newtheorem*{solution}{Solution}
\newtheorem*{resources}{Resources}

\newcommand{\name}[1]{\noindent\textbf{Name: #1}}
\newcommand{\honor}{\noindent On my honor, as an Aggie, I have neither
  given nor received any unauthorized aid on any portion of the
  academic work included in this assignment. Furthermore, I have
  disclosed all resources (people, books, web sites, etc.) that have
  been used to prepare this homework. \\[1ex]
 \textbf{Signature:} \underline{\hspace*{5cm}} }

 \newcommand{\checklist}{\noindent\textbf{Checklist:}
\begin{compactitem}[$\Box$] 
\item Did you add your name? 
\item Did you disclose all resources that you have used? \\
(This includes all people, books, websites, etc. that you have consulted)
\item Did you sign that you followed the Aggie honor code? 
\item Did you solve all problems? 
\item Did you submit (a) the pdf file derived from your latex source file
  of your homework?
\item Did you submit (b) a hardcopy of the pdf file in class? 
\end{compactitem}
}

\newcommand{\problemset}[1]{\begin{center}\textbf{Problem Set
      #1}\end{center}}
\newcommand{\duedate}[2]{\begin{quote}\textbf{Due dates:} Electronic
    submission of .tex and .pdf files of this homework is due on
    \textbf{#1} on ecampus, a signed paper copy of the pdf file is due
    on \textbf{#2} at the beginning of class. \end{quote} }

\newcommand{\N}{\mathbf{N}}
\newcommand{\R}{\mathbf{R}}
\newcommand{\Z}{\mathbf{Z}}


\begin{document}
\problemset{3}
\duedate{2/10/2017 before 10:00am}{2/10/2017}
\name{Jonathan Janzen}
\begin{resources}
Textbook, Lectures/slides, talking to classmates.
\end{resources}
\honor

\newpage
Make sure that you describe all solutions in your own words, even
though many of the exercises were part of team explorations!

Read chapters 4 and 30 in our textbook. 

\begin{problem}
\begin{compactenum}[(a)]
\item (10 points) Exercise 30.1-4 on page 905 in our textbook. 
\item (10 points) Exercise 30.1-5 on page 906 in our textbook. 
\item (10 points) Exercise 30.2-1 on page 914 in our textbook. 
\end{compactenum}
\end{problem}

\begin{solution}
\begin{compactenum}[(a)]
\item With $n-1$ point-value pairs, we can construct a unique polynomial with a degree of $n-1$. By some arbitrary point into it, it will then contain $n$ point-value pairs and the degree will increment to be $n$. There is no way to have $n-1$ points define a a polynomial with degree $n$ because $n-1$ points would define a polynomial with degree $n-1$.
\item The formula defined by (30.5) can be rewritten as:
$$A(x)=\Sigma_{k=0}^{n-1} y_k \Pi_{j \neq k} \frac{x-x_j}{x_k-x_j}$$
This formula will execute the inner loop for each $x$ and execute the outer loop for each $x$ therefore the total execution time will be $O(n^2)$
\item The Corollary $\omega_n^{n/2} = w_2 = -1$ can be proved by:
$$ \omega_n^{n/2} = (e^{2\pi i / n})^{n/2} = e^{\pi i } = -1$$
\end{compactenum}
\end{solution}

\begin{problem} 
\begin{compactenum}[(a)]
\item (10 points) Suppose that you are given a polynomial 
$$ A(x) = \sum_{k=0}^{n-1} a_k x^k.$$ 
The input to the FFT of length $n$ is given by an array containing the coefficients
$(a_0,\ldots, a_{n-1})$. Describe the output of the FFT in terms of
the polynomial $A(x)$. 

\item (10 points) Let $\omega$ be a primitive $n$th root of unity. 
The fast Fourier transform implements the multiplication with
  the matrix 
$$ F = (\omega^{ij})_{i,j\ in [0..n-1]}.$$
Show that the inverse of the $F$ is given by 
$$ F^{-1} = \frac{1}{n}  (\omega^{-jk})_{j,k\ in [0..n-1]}$$
[Hint: $x^n-1= (x-1)(x^{n-1}+\cdots + x + 1),$ so any power
$\omega^\ell\neq 1$  must be a root of $x^{n-1}+\cdots + x + 1$.  ]  
Thus, the inverse FFT, called IFFT, is nothing but the FFT using
$\omega^{-1}$ instead of $\omega$, and multiplying the result with
$1/n$. 
\item (10 points) Describe how to do a polynomial multiplication using the FFT and
  IFFT for polynomials $A(x)$ and $B(x)$ of degree $\le n-1$. Make
  sure that you describe the length of the FFT and IFFT needed for
  this task. 
\item (15 points) How can you modify the polynomial multiplication algorithm based
  on FFT and IFFT to do multiplication of long integers in base 10?
  Make sure that you take care of carries in a proper way. 
\item (5 points) What kind of problems can occur in the previous
  approach to multiply long integers and how would you overcome them? 
\end{compactenum}
\end{problem}
\begin{solution}
\begin{compactenum}[(a)]
\item The output will be a series of y-values matching the inputs for the positive and negative roots of unity:
$$(A(\omega^0),A(\omega^1),...,A(\omega^{n/2}),A(-\omega^0),A(-\omega^1),...,A(-\omega^{n/2}))$$
\item Since we know that the Fast Fourier Transform performs the same operation as $F$ and we know that the IFFT operation is the same as FFT except that we provide negative $\omega$ values and divide the equation by $n$ it is safe to assume that $F^{-1}$ will need the same modifications as IFFT which leads to the definition: $F^{-1} = \frac{1}{n}(w^{-jk})_{j,k \in [0..n-1]}$
\item Apply FFT on $A(x)$ and $B(x)$ separately to get $\alpha(x)$ and $\beta(x)$ (respectively) which operates in $O(n\log n)$ for each function. Now that they're in point-value form we can multiply them in $O(n)$ time as $\alpha(x)\beta(x) = \gamma(x)$. We can now use IFFT to convert $\gamma(x)$ into the coefficient form $C(x)$ in $O(n\log n)$ time. Overall the problem takes $O(n \log n)$ time based on the following summation of all sub-steps:
$$O(n \log n) + O(n \log n) + O(n) + O(n \log n) + O(n \log n) = O(n \log n)$$
\item Let $n,m$ be long integers written in base 10. Each integer can be written as a sum of coefficients multiplied by 10 to a power matching its decimal place:
$$n = n_0 * 10^0 + n_1 * 10^1 + ...$$
We can then pretend that this is a polynomial by substituting $x$ in for $10$: 
$$A(x) = n_0 * x^0 + n_1 * x^1 + ...$$
Then we can let $n = A(10)$. The same process can be completed for $m$ resulting in: $m = B(10)$.
We can apply the Fast-Fourier Transform method to $A(x),B(x)$ and then multiply them together to get a new polynomial $C(x)$. Then we can apply the inverse Fast-Fourier Transform to get the coefficient form of $C(x)$ and we can evaluate $C$ at $x=10$ to get the value of the multiplication of $n$ and $m$.
\item The problem with this method is that it doesn't handle carrying correctly. Using the naming conventions from the previous example, $C(x)$ may have coefficients that are greater than 9, which means that when evaluating the polynomial at 10 to get an answer it will result in an invalid answer.
\end{compactenum}
\end{solution}

\begin{problem} (20 points) You overhear a conversation where someone
  mentions that Morgenstern proved an $\Omega(n\log n)$ lower bound on
  the fast Fourier transform and someone else mentions that a group of
  MIT researchers found in 2012 a faster than fast Fouier transform
  that is $o(n\log n)$. These two comments seem to contradict each
  other. Do your research and find out what Morgenstern really proved
  and under what circumstances the MIT algorithm can improve on the
  FFT.
\end{problem}
\begin{solution}
The MIT makes improvements on Sparse Fast Fourier Transform (i.e. when coefficients are small or near to zero). The Morgenstern proof refers to the lower bound using linear algorithms. The problems are separate.
\end{solution}

Discussions on ecampus are always encouraged, especially to clarify
concepts that were introduced in the lecture. However, discussions of
homework problems on piazza should not contain spoilers. It is okay to
ask for clarifications concerning homework questions if needed. 
\medskip



\goodbreak
\checklist
\end{document}
