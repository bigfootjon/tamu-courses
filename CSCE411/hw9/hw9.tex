\documentclass{article}
\usepackage{amsmath,amssymb,amsthm,latexsym,paralist}

\theoremstyle{definition}
\newtheorem{problem}{Problem}
\newtheorem*{solution}{Solution}
\newtheorem*{resources}{Resources}

\newcommand{\name}[1]{\noindent\textbf{Name: #1}}
\newcommand{\honor}{\noindent On my honor, as an Aggie, I have neither
  given nor received any unauthorized aid on any portion of the
  academic work included in this assignment. Furthermore, I have
  disclosed all resources (people, books, web sites, etc.) that have
  been used to prepare this homework. \\[1ex]
 \textbf{Signature:} \underline{\hspace*{5cm}} }

 \newcommand{\E}{\textup{E}}
 
\newcommand{\checklist}{\noindent\textbf{Checklist:}
\begin{compactitem}[$\Box$] 
\item Did you add your name? 
\item Did you disclose all resources that you have used? \\
(This includes all people, books, websites, etc. that you have consulted)
\item Did you sign that you followed the Aggie honor code? 
\item Did you solve all problems? 
\item Did you submit (a) your latex source file and (b) the resulting pdf file
  of your homework?
\item Did you submit (c) a hardcopy of the pdf file in class? 
\end{compactitem}
}

\newcommand{\problemset}[1]{\begin{center}\textbf{Problem Set
      #1}\end{center}}
\newcommand{\duedate}[2]{\begin{quote}\textbf{Due dates:} Electronic
    submission of this homework is due on \textbf{#1} on ecampus, a
    signed paper copy of the pdf file is due on \textbf{#2} at the
    beginning of class. \end{quote} }

\newcommand{\N}{\mathbf{N}}
\newcommand{\R}{\mathbf{R}}
\newcommand{\Z}{\mathbf{Z}}


\begin{document}
\problemset{9}
\duedate{Monday 4/3/2017 before 10:00am}{4/3/2017}
\name{Jonathan Janzen}
\begin{resources} Textbook, Perusall readings, lecture notes, talking to classmates
\end{resources}
\honor

\begin{problem}[10 points] Solve Exercise 2.7 of the randomized
  algorithms lecture notes (on perusall.com). 
\end{problem}
\begin{solution}
$X = x$ is shorthand for $\{z \in \Omega | X(z) \in A\}$ where $A = {x}$. Since $A$ is a singleton, this can be rewritten as $\{z \in \Omega | X(z) = x\}$. This set constructor will produce a set of outcomes that are equal to $x$. A set of outcomes is an event, so $X = x$ produces an event in $\mathcal{F}$.
\end{solution}

\begin{problem}[10 points] Solve Exercise 2.8 of the randomized
  algorithms lecture notes (on perusall.com). 
\end{problem}
\begin{solution}
Given that the distribution function is defined as $F_X(x) = Pr[X \leq x]$, it follows that $X \leq x$. Since we are trying to prove $x \leq y$ this can be extended to $X \leq x \leq y$. This would then imply that $Pr[X \leq x] \leq Pr[X \leq y]$ or:
$$ F_X(x) \leq F_X(y) $$
In other words, the distribution function is defined as the probability that $X$ is less than $x$, and if $y$ is greater than $x$ then there must be more things that are less than $y$ (or at least the same amount).
\end{solution}

\begin{problem}[20 points] Solve Exercise 2.9 of the randomized
  algorithms lecture notes (on perusall.com). 
\end{problem}
\begin{solution}
Since $h(X)$ is a real-to-real function it follows that it can be considered a random variable, here denoted $Y$ since random variables must take a $\Omega$ as input and produce real numbers as outputs, which $h(X)$ does:
$$ Pr[Y \geq t] \leq \frac{\E[Y]}{t} $$
Furthermore, since $h(x)$ is defined as non-negative (i.e. $h(x) \geq 0$ for all $x \in \R$), it follows that $h(x) = |h(x)|$ which means we have:
$$ Pr[|Y| \geq t] \leq \frac{\E[|Y|]}{t} $$
Which is proved by the Markov Inequality. Therefore by substitution ($|Y| = h(x)$ and $Y = h(x)$) we can obtain the original form of the problem, which we now know is an instance of Markov's Inequality:
$$ Pr[h(x) \geq t] \leq \frac{\E[h(x)]}{t} $$
\end{solution}


\begin{problem}[10 points]
Suppose that the running time of a randomized algorithm is modeled by
the random variable $X$. You have determined the expected running time
$\E[X]$. Use Markov's inequality to bound the probability that an
execution of the randomized algorithm exceeds
$(1+\epsilon)\E[X]$. 
\end{problem}
\begin{solution}
Assuming that $X$ is non-negative, we can rewrite $(1 + \epsilon)\E[X]$ to find a value for $t$:
$$ \frac{\E[X]}{\frac{1}{1+\epsilon}} \therefore t = \frac{1}{1+\epsilon} $$
Therefore, the bounds given by Markov's Inequality are as follows:
$$ Pr[X \geq \frac{1}{1+\epsilon}] \leq \frac{\E[X]}{\frac{1}{1+\epsilon}} $$
\end{solution}

\begin{problem}[10 points]
Suppose that the running time of a randomized algorithm is modeled by
the random variable $X$. Suppose that you know the expected running time
$\E[X]$. You know that your algorithm has a worst case running time
that exceeds $\E[X]$ by a significant margin. Therefore, you decide
to stop the execution whenever it exceeds $(1+\epsilon)\E[X]$ steps,
and restart the algorithm. You will repeat the execution of the
algorithm at most $t$ times. Denote by $X_k$ the random variable
modeling running time of the $k$-th try. Determine the probability
that the randomized algorithm exceeds  $(1+\epsilon)\E[X]$ steps in
all $t$ repetitions. This probability models the failure probability
of this algorithm.
\end{problem}
\begin{solution}
This question can be expressed as a probability with:
$$ Pr[X \geq (1+\epsilon)\E[X]] $$
Which is the left hand sign of the matching Markov Inequality:
$$ Pr[X \geq (1+\epsilon)\E[X]] \leq \frac{\E[X]}{(1+\epsilon)\E[X]} $$
Simplifying:
$$ Pr[X \geq (1+\epsilon)\E[X]] \leq \frac{1}{(1+\epsilon)} $$
This is the upper bound on the probability that execution exceeds $(1+\epsilon)\E[X]$ steps.
\end{solution}

\begin{problem} [20 points] How should we choose the number of repetitions $t$
  such that the randomized algorithm has with probability $1-1/n$ at
  least one run such that $X\le (1+\epsilon)\E[X]$ among the $t$ runs.
\end{problem}
\begin{solution}
We are searching for $Pr[X \leq (1+\epsilon)\E[X]]$, we can solve this using the complement as follows:
$$ Pr[X \leq (1+\epsilon)\E[X]] = 1 - Pr[X \geq (1+\epsilon)\E[X]] $$
The right hand sign can be represented as a Markov Inequality:
$$ Pr[X \geq (1+\epsilon)\E[X]] \leq \frac{\E[X]}{(1+\epsilon)\E[X]} $$
We can use the complement of the original probability to find the probability for the complement:
$$ 1 - (1 - \frac{1}{n}) = 1 - 1 + \frac{1}{n} = \frac{1}{n} $$
We can plug this in for the probability that execution exceeds $(1+\epsilon)\E[X]$ steps in the Markov Inequality:
$$ \frac{1}{n} \leq \frac{1}{(1+\epsilon)} $$
$$ \epsilon \leq n -1 $$
We can use this to solve for $t = (1+\epsilon)E[X]$ in the Markov Inequality:
$$ t = (1+ \epsilon)E[X] $$
\end{solution}

\begin{problem}[20 points] How many people need to be in a room such that two of
  those   people share a birthday with
  probability of $97\%$.  We assume that birthdays are uniformly
  distributed. You need to derive your result. 
\end{problem}
\begin{solution}
Slide 5 of random algorithms part 5 gives the following:
$$m \geq \frac{1}{2}(1 + \sqrt{1-8n*\text{ln}(\delta)})$$
where $p_{uni} \geq 1 - \delta$, $m$ is the number of people, $n$ is the length of a year. Plugging in $p_{uni}=.97$ we get $\delta = .03$ and we can plug that and $n=365$ into the inequality:
$$ m \geq \frac{1}{2}(1 + \sqrt{1-2920*\text{ln}(.03)}) $$
$$ m \geq 51.09$$
So we need at least 52 people to say with a 97\% certainty that 2 or more people share a birthday.
\end{solution}



\goodbreak
\checklist
\end{document}
